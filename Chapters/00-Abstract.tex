\thispagestyle{plain}
%\chapter*{Resumo}
%
%\guiainfo{Na secção \textit{Resumo}, apresente um resumo conciso do seu projeto, destacando os pontos principais. Comece com uma breve declaração do problema ou objetivo, seguido de uma descrição da sua abordagem ou metodologia. Resuma os principais resultados ou conclusões, salientando a sua importância ou implicações. Conclua com uma ou duas frases sobre a contribuição global ou o impacto do seu trabalho. O resumo deve ser claro e conciso, idealmente com 150-250 palavras, para que os leitores compreendam rapidamente o seu trabalho e a sua importância.}
%
%\keywordspt{Palavra-Chave A, Palavra-Chave B, Palavra-Chave C.}
%
%\MediaOptionLogicBlank

\pdfbookmark[1]{Abstract}{abstract}
\chapter*{Abstract}

% In the \textit{Abstract} section, provide a concise summary of your project, highlighting the key points. Begin with a brief statement of the problem or objective, followed by a description of your approach or methodology. Summarise the main results or findings, emphasising their significance or implications. Conclude with a sentence or two on the overall contribution or impact of your work. Keep the abstract clear and focused, ideally within 150-250 words, to give readers a quick understanding of your research and its importance.

This thesis addresses the premature convergence limitation of Particle Swarm Optimization (PSO), especially in complex, high-dimensional problems. It introduces and evaluates novel informed diversity-enhancing strategies to improve PSO's exploration capabilities. The proposed methodology involves a family of mechanisms based on problem-specific landmarks (best/worst solutions) that apply attraction or repulsion forces within velocity updates. These strategies are categorized into opposing-best (repulsion), attraction-to-worst (negative learning), and opposing-worst (reverse learning) behaviors, forming both single-role algorithms and three multi-hybrid PSO par\-a\-digms: disjoint-role, component-specific, and fully flexible. Rigorous empirical testing on diverse benchmark functions up to 1000 dimensions demonstrates that these strategies, particularly multi-hybrid variants like component-specific Hy\-brid\-Par\-tial\-Dis
-jointPSO, significantly outperform standard PSO and simple perturbation methods in solution quality, convergence, and robustness. Variants employing attraction-to-worst strategies also showed strong, consistent performance. This research establishes that informed, role-based diversity mechanisms are, in most cases, more effective than random perturbations, offering scalable and robust PSO enhancements for complex optimization by better balancing exploration and exploitation.



% \keywordsen{Keyword A, Keyword B, Keyword C.}

\keywordsen{Evolutionary Computation, Swarm Intelligence, Hybrid Metaheuristics, Particle Swarm Optimization (PSO)}


\MediaOptionLogicBlank