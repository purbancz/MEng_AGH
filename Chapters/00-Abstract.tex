\thispagestyle{plain}
%\chapter*{Resumo}
%
%\guiainfo{Na secção \textit{Resumo}, apresente um resumo conciso do seu projeto, destacando os pontos principais. Comece com uma breve declaração do problema ou objetivo, seguido de uma descrição da sua abordagem ou metodologia. Resuma os principais resultados ou conclusões, salientando a sua importância ou implicações. Conclua com uma ou duas frases sobre a contribuição global ou o impacto do seu trabalho. O resumo deve ser claro e conciso, idealmente com 150-250 palavras, para que os leitores compreendam rapidamente o seu trabalho e a sua importância.}
%
%\keywordspt{Palavra-Chave A, Palavra-Chave B, Palavra-Chave C.}
%
%\MediaOptionLogicBlank

\pdfbookmark[1]{Abstract}{abstract}
\chapter*{Abstract}

% In the \textit{Abstract} section, provide a concise summary of your project, highlighting the key points. Begin with a brief statement of the problem or objective, followed by a description of your approach or methodology. Summarise the main results or findings, emphasising their significance or implications. Conclude with a sentence or two on the overall contribution or impact of your work. Keep the abstract clear and focused, ideally within 150-250 words, to give readers a quick understanding of your research and its importance.

This thesis proposes and rigorously evaluates a novel family of diversity-enhancing strategies for Particle Swarm Optimization (PSO) aimed at addressing premature convergence and insufficient exploration—two persistent limitations of the canonical algorithm. The approach introduces both single-role variants, such as opposing-best and negative learning strategies, and advanced multi-hybrid frameworks that systematically combine diverse behavioral roles within dedicated sub-swarms or probabilistically within individual particles. Extensive empirical experiments on a broad suite of continuous benchmark problems, tested at dimensions up to 1000, demonstrate that these informed diversity mechanisms consistently outperform both the canonical PSO and a standard perturbation-based variant. In particular, the HybridPartialDisjointPSO and Defeatist-based extensions deliver superior solution quality, robust convergence behavior, and excellent scalability. The findings confirm that incorporating problem-informed exploration via targeted role design is markedly more effective than naive random perturbations. This work provides a robust, general-purpose enhancement to PSO and lays the groundwork for future extensions toward adaptive, self-configuring swarm frameworks capable of tackling increasingly complex and dynamic optimization tasks.

% \keywordsen{Keyword A, Keyword B, Keyword C.}

\keywordsen{Evolutionary Computation, Swarm Intelligence, Hybrid Metaheuristics, Particle Swarm Optimization (PSO)}


\MediaOptionLogicBlank